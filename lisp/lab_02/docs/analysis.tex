\textbf{Задание №1}

Составить диаграмму вычисления следующих выражений:
\begin{enumerate}
    \item (equal 3 (abs - 3))
    \item (equal (+ 1 2) 3)
    \item (equal (* 4 7) 21)
    \item (equal (* 2 3) (+ 7 2))
    \item (equal (- 7 3) (* 3 2))
    \item (equal (abs (- 2 4)) 3))
\end{enumerate}

\imgScale{0.5}{task_1_1}{Диаграмма №1}
\imgScale{0.5}{task_1_2}{Диаграмма №2}
\imgScale{0.5}{task_1_3}{Диаграмма №3}
\imgScale{0.5}{task_1_4}{Диаграмма №4}
\imgScale{0.5}{task_1_5}{Диаграмма №5}
\imgScale{0.5}{task_1_6}{Диаграмма №6}
\FloatBarrier


\textbf{Задание №2}

Написать функцию, вычисляющую гипотенузу прямоугольного
треугольника по заданным катетам и составить диаграмму её вычисления.

\begin{center}
    \captionsetup{justification=raggedright,singlelinecheck=off}
    \begin{lstlisting}[label=lst:task_2,caption=Код функции]
    (defun gip (a b)
    (sqrt (+ (* a a) (* b b))))
\end{lstlisting}
\end{center}

\imgScale{0.5}{task_2}{Диаграмма вычисления функции gip}
\FloatBarrier

\clearpage
\textbf{Задание №3}

Каковы результаты вычисления следующих выражений? (объяснить возможную ошибку и варианты ее устранения)
\begin{enumerate}
    \item (list 'a c) --- The variable C is unbound.
    \item (cons 'a (b c)) --- The variable C is unbound. The function COMMON-LISP-USER::B is undefined.
    \item (cons 'a '(b c)) --- (a b c)
    \item (caddr (1 2 3 4 5)) --- Execution of a form compiled with errors. Form: (1 2 3 4 5). Compile-time error: illegal function call.
    \item (cons 'a 'b 'c) --- invalid number of arguments: 3.
    \item (list 'a (b c)) --- The variable C is unbound. The function COMMON-LISP-USER::B is undefined.
    \item (list a '(b c)) --- The variable A is unbound.
    \item (list (+ 1 '(length '(1 2 3)))) --- The value (LENGTH '(1 2 3)) is not of type NUMBER when binding SB-KERNEL::Y.
\end{enumerate}

Объяснение ошибок и способы их устранения.
\begin{enumerate}
    \item The variable C is unbound (за 'c' не закреплено никакого значения). \textit{Устранение}: задать значение 'c'.
    \item The function COMMON-LISP-USER::B is undefined. (не определена функция 'b'). \textit{Устранение}: определить функцию 'b'.
    \item Illegal function call (некорректный вызов функции, так как в качестве названия функции написана 1). \textit{Устранение}: использовать 'quote' перед (1 2 3 4 5).
    \item Invalid number of arguments: 3 (Некорректное количество аргументов функции cons). \textit{Устранение}: использовать только два аргумента.
    \item The variable A is unbound (за 'a' не закреплено никакого значения). \textit{Устранение}: задать значение 'a'.
    \item The value (LENGTH '(1 2 3)) is not of type NUMBER when binding SB-KERNEL::Y (чтобы можно было сложить 1 со вторым значением, оно должно быть числовым, но так как перед 'length' стоит 'quote', функция 'length' не выполняется, а остается как часть списка). \textit{Устранение}: убрать 'quote' перед 'length'.
\end{enumerate}

\textbf{Задание №4}

Написать функцию longer\_then от двух списков-аргументов, которая возвращает Т, если первый аргумент имеет большую длину.

\begin{center}
    \captionsetup{justification=raggedright,singlelinecheck=off}
    \begin{lstlisting}[label=lst:task_2,caption=Код функции]
    (defun longer_than (a b)
    (> (length a) (length b)))
\end{lstlisting}
\end{center}

\textbf{Задание №5}

Каковы результаты вычисления следующих выражений?
\begin{enumerate}
    \item (cons 3 (list 5 6)) --- (3 5 6)
    \item (cons 3 '(list 5 6)) --- (3 list 5 6)
    \item (list 3 'from 9 'lives (-- 9 3)) --- (3 from 9 lives 6)
    \item (+ (length for 2 too)) --- The variable FOR is unbound
    \item (car '(21 22 23)))  --- 21
    \item (cdr '(cons is short for ans)) --- (is short for ans)
    \item (car (list one two)) --- The variable ONE is unbound
    \item (car (list 'one 'two)) --- one
\end{enumerate}

\textbf{Задание №6}

Дана функция:
\begin{center}
    \captionsetup{justification=raggedright,singlelinecheck=off}
    \begin{lstlisting}[label=lst:task_2,caption=Код функции]
    (defun mystery (x) 
    (list (second x) (first x)))
\end{lstlisting}
\end{center}
Какие результаты вычисления следующих выражений?

Все эти выражения выдадут ошибку, если за соответствующими атомами не закреплены никакие значения. Чтобы исправить именно эту ошибку, можно использовать 'quote' перед каждым таким атомом.
\begin{enumerate}
    \item (mystery (one two)) --- The variable TWO is unbound. The function COMMON-LISP-USER::ONE is undefined. Исправление: (mystery '(one two)), тогда вывод --- (two one).
    \item (mystery one 'two)) --- The variable ONE is unbound. 
    \item (mystery (last one two)) --- The variable ONE is unbound. The variable TWO is unbound. Исправление: использовать функцию 'list' (mystery (last one two)), вывод --- (two one).
    \item (mystery free) --- The value FREE is not of type LIST.
\end{enumerate}

\textbf{Задание №7}

Написать функцию, которая переводит температуру в системе Фаренгейта
температуру по Цельсию (defum f-to-c (temp) …).

Формулы: $c = 5/9*(f-32.0)$; $f= 9/5*c+32.0$.
\begin{center}
    \captionsetup{justification=raggedright,singlelinecheck=off}
    \begin{lstlisting}[label=lst:task_2,caption=Код функции f-to-c]
    (defun f-to-c (temp)
    (* (/ 5 9) (- temp 32.0)))
\end{lstlisting}
\end{center}

Как бы назывался роман Р.Брэдбери <<+451 по Фаренгейту>> в системе по Цельсию?

\textit{Ответ}: 232 градуса по Цельсию.\newline

\textbf{Задание №8}

Каковы результаты вычисления следующих выражений?
\begin{enumerate}
    \item (list 'cons t NIL) --- (cons T NIL)
    \item (eval (list 'cons t NIL)) --- (T)
    \item (eval (eval (list 'cons t NIL))) --- The function COMMON-LISP:T is undefined.
    \item (apply \#cons "(t NIL)) --- некорректный синтаксис, если\newline исправить "\ на ' и после \# написать ', то результатом будет (T).
    \item (eval NIL) --- NIL
    \item (list 'eval NIL) --- (EVAL NIL)
    \item (eval (list 'eval NIL)) --- NIL
\end{enumerate}
