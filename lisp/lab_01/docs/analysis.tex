\chapter{Задание №1}

Представить следующие списки в виде списочных ячеек:
\begin{enumerate}
    \item '(open close halph)
    \item '((open1) (close2) (halph3))
    \item '((one) for all (and (me (for you))))
    \item '((TOOL) (call))
    \item '((TOOL1) ((call2)) ((sell)))
    \item '(((TOOL) (call)) ((sell)))
\end{enumerate}

\imgScale{0.35}{falp_lab_01_1}{Решение задания №1}


\chapter{Задание №2}

Используя только функции CAR и CDR, написать выражения,
возвращающие:
\begin{itemize}
    \item второй;
    \item третий;
    \item четвертый элементы заданного списка.
\end{itemize}

\begin{center}
    \captionsetup{justification=raggedright,singlelinecheck=off}
    \begin{lstlisting}[label=lst:task_2,caption=Решение задания №2]
    (car (cdr '(a b c d e f)))
    (car (cdr (cdr '(a b c d e f))))
    (car (cdr (cdr (cdr '(a b c d e f)))))
\end{lstlisting}
\end{center}


\chapter{Задание №3}

Что будет в результате вычисления выражений?

\begin{enumerate}
    \item (CAADR '((blue cube) (red pyramid)))
    \item (CDAR '((abc) (def) (ghi)))
    \item (CADR '((abc) (def) (ghi)))
    \item (CADDR '((abc) (def) (ghi)))
\end{enumerate}

Результаты:

\begin{enumerate}
    \item red
    \item Nil
    \item (def)
    \item (ghi)
\end{enumerate}

\chapter{Задание №4}

Напишите результат вычисления выражений и объясните как он получен:

\begin{enumerate}
    \item (list 'Fred 'and 'Wilma) --- (Fred and Wilma)
    \item (cons 'Fred '(and Wilma)) --- (Fred and Wilma)
    \item (list 'Fred '(and Wilma)) --- (Fred (and Wilma))
    \item (cons 'Fred '(Wilma)) --- (Fred Wilma)
    \item (cons Nil Nil) --- (Nil)
    \item (list Nil Nil) --- (Nil Nil)
    \item (cons T Nil) --- (T)
    \item (list T Nil) --- (T Nil)
    \item (cons Nil T) --- (Nil . T)
    \item (list Nil T) --- (Nil T)
    \item (list Nil) --- (Nil)
    \item (cons T (list Nil)) --- (T Nil)
    \item (cons '(T) Nil) --- ((T))
    \item (list '(T) Nil) --- ((T) Nil)
    \item (list '(one two) '(free temp)) --- ((one two) (free temp))
    \item (cons '(one two) '(free temp)) --- ((one two) free temp)
\end{enumerate}

\chapter{Задание №5}

Написать лямбда-выражение и соответствующую функцию.
\begin{enumerate}
    \item Написать функцию (f arl ar2 ar3 ar4), возвращающую список: ((arl ar2) (ar3 ar4)).
    \item Написать функцию (f arl ar2), возвращающую ((arl) (ar2)).
    \item Написать функцию (f arl), возвращающую (((arl))).
    \item Представить результаты в виде списочных ячеек.
\end{enumerate}

\begin{center}
    \captionsetup{justification=raggedright,singlelinecheck=off}
    \begin{lstlisting}[label=lst:d_l_rec,caption=Решение задания №5]
    ; #1
    ; lambda
    (lambda (ar1 ar2 ar3 ar4) (list (list ar1 ar2) (list ar3 ar4)))
    
    ; func
    (defun f1 (ar1 ar2 ar3 ar4) (list (list ar1 ar2) (list ar3 ar4)))
    
    ; #2
    ; lambda
    (lambda (ar1 ar2) (list (list ar1) (list ar2)))
    
    ;func
    (defun f2 (ar1 ar2) (list (list ar1) (list ar2)))
    
    ; #3
    ; lambda
    (lambda (ar1) (list (cons (list ar1) Nil)))
    
    ; func
    (defun f3 (ar1) (list (cons (list ar1) Nil)))
\end{lstlisting}
\end{center}

\imgScale{0.5}{falp_lab_01_5}{Решение задания №5}
